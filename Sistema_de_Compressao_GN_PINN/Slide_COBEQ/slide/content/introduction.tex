\section{Introdução}

\begin{frame}{Contexto e Objetivo do Trabalho}
    \scriptsize
    \textbf{Contexto Histórico na Indústria do Petróleo:}
    
    \begin{itemize}
        \item \textbf{Importância do Transporte de Gás Natural:}
        movimentar o gás até consumidores finais (indústrias, usinas, centros urbanos) é um processo essencial, porém com alto custo operacional.

        \item \textbf{Limitações dos Modelos Tradicionais:}
        métodos numéricos apresentam elevado tempo computacional (\cite{MARFATIA2022100030}), enquanto modelos aproximados sacrificam precisão.

        \item \textbf{Avanços com Redes Neurais:}
        desde os anos 90, redes neurais têm sido aplicadas no setor (\cite{Mohaghegh1996}), buscando equilíbrio entre precisão e eficiência.

        \item \textbf{Physics-Informed Neural Networks (PINNs):}
        propostas por \cite{Raissi2017}, integram dados experimentais às leis físicas, aumentando a precisão e reduzindo o tempo de simulação.
    \end{itemize}

    \vspace{0.4cm}

    \begin{block}{Objetivo Principal}
        Construir uma \textbf{PINN} (Physics Informed Neural Network) que modele o comportamento dinâmico de um sistema de compressão de gás natural.
    \end{block}
\end{frame}

