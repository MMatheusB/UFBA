\section{Introdução}

\begin{frame}{Objetivo}
    \frametitle{Motivação do Trabalho}
    \begin{block}{Objetivo Principal}
        Construir uma \textbf{PINN}(Physics Informed Neural Network) que modele o comportamento dinâmico de um sistema de compressão de gás natural.
    \end{block}
    \begin{block}{Especificações}
        \begin{itemize}
            \item Basear-se em equações físicas para maior precisão do modelo criado.
            \item Comparar resultados com o integrador da biblioteca opensource casADi (IDAS).
        \end{itemize}
    \end{block}
    \end{frame}

\begin{frame}{Contexto Histórico do Tema na Indústria do Petróleo}
    \begin{itemize}
        \item \textbf{Importância do Transporte de Gás Natural}
        \begin{itemize}
            \item Movimento do gás até consumidores finais (áreas urbanas, indústrias e usinas). 
            \item Processo crucial e com alto custo operacional. 
        \end{itemize}

        \item \textbf{Desafios nos Modelos Tradicionais} 
        \begin{itemize}
            \item Métodos numéricos possuem alto tempo computacional (\cite{MARFATIA2022100030}). 
            \item Modelos aproximados sacrificam precisão para ganhar eficiência computacional. 
        \end{itemize}

        \item \textbf{Avanço com Redes Neurais} 
        \begin{itemize}
            \item Década de 90: Redes neurais começam a ganhar relevância no setor de petróleo e gás (\cite{Mohaghegh1996}). 
            \item Melhor equilíbrio entre eficiência e precisão nos resultados. 
        \end{itemize}

        \item \textbf{Physics-Informed Neural Networks (PINNs)} 
        \begin{itemize}
            \item Introduzidas em 2017 (\cite{Raissi2017}). 
            \item Integram dados experimentais e restrições físicas. 
            \item Maior precisão e menor tempo computacional. 
        \end{itemize}
    \end{itemize}
\end{frame}
