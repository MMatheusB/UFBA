\section{Estrutura da Rede Neural Artificial}

% Equações Governantes
\begin{frame}{Sistema de Compressão e Gás Natural}
    \begin{columns}[T] % [T] alinha as colunas pelo topo
        % Coluna da esquerda - Tabela
        \begin{column}{0.48\textwidth}
        \begin{figure}
            \centering
            \includegraphics[width=1.1\linewidth]{figures/compressao.png}
            \caption{Sitema de Compressão retirado de \cite{Meira2022}}
            \label{fig:enter-label}
        \end{figure}
        \end{column}

        % Coluna da direita - Composição e EoS
        \begin{column}{0.52\textwidth}
            \scriptsize
        \begin{block}{Composição do gás}
                
            O gás natural utilizado é rico em metano, com composição baseada em \cite{Chaczykowski2009}:
            
            \vspace{0.12cm}
            \begin{itemize}  
                \item CH\textsubscript{4}: 98,34\% \quad C\textsubscript{2}H\textsubscript{6}: 0,61\%
                \item C\textsubscript{3}H\textsubscript{8}: 0,15\% \quad iC\textsubscript{4}H\textsubscript{10}: 0,03\%
                \item nC\textsubscript{4}H\textsubscript{10}: 0,03\% \quad CO\textsubscript{2}: 0,80\%
                \item Traços de: iC\textsubscript{5}H\textsubscript{12}, nC\textsubscript{5}H\textsubscript{12}, N\textsubscript{2}
            \end{itemize}
        \end{block}
            \vspace{0.1cm}
            A equação de estado de \cite{Soave1972} foi utilizada para modelar o comportamento termodinâmico do gás:

            \[
            P = \frac{R T}{V - b} - \frac{a(T)}{V(V + b)}
            \]

            com:
            \begin{itemize}
                \item \( a(T) \): fator de correção das forças intermoleculares
                \item \( b \): correção do volume molecular
            \end{itemize}
        \end{column}
    \end{columns}
\end{frame}



\begin{frame}{Equações e Variáveis do Modelo de \cite{Meira2022} }
    \scriptsize
    \textbf{Equações que descrevem a dinâmica do sistema:}

    \begin{multicols}{2}
    % Coluna da esquerda: equações diferenciais
    \begin{align}
        \frac{d\dot{m}}{dt} &= \frac{A_1}{L_c}(P_2 - P_P) \tag{1} \\
        \frac{dV_P}{dt} &= -\frac{V_P^2}{v_{PM}} \left( \dot{m} - \alpha k_v \sqrt{P_P - P_{\text{out}}} \right) \tag{2} \\
        \frac{dT_P}{dt} &= 
        \begin{aligned}[t]
            &\frac{V_P \dot{m}}{v_P M} \left( \frac{h_c - h_p}{C_V} \right) + \\
            + &\frac{R_a T_P}{C_V} \left[ T_P \left( \frac{\partial Z_P}{\partial T} \right)_{V_P} + Z_P \right]
            \frac{V_P}{v_P M} \left( \dot{m} - \alpha k_v \sqrt{P_P - P_{\text{out}}} \right)
        \end{aligned} \tag{3} \\
            0 &= f(x, z, u) \tag{4} \qquad \text{(equações algébricas)}
    \end{align}


    \columnbreak
    % Coluna da direita: variáveis algébricas
    \begin{minipage}{\linewidth}
        \textbf{Variáveis estimadas pelo modelo:}
        \begin{itemize}
            \item \textcolor{blue}{\( \dot{m} \)}: vazão mássica
            \item \textcolor{blue}{\( T_P \)}, \( V_P \), \textcolor{blue}{\( P_P \)}: variáveis no plenum
            \item \textcolor{blue}{\( P_2 \)}, \textcolor{blue}{\( T_2 \)}, \( V_2 \): saída do compressor
            \item \( T_{2s} \), \( V_{2s} \): pós-compressão isentrópica
            \item \( T_{imp} \), \( V_{imp} \): impelidor
            \item \( T_{dif} \), \( V_{dif} \): difusor
        \end{itemize}

        \vspace{1.5cm}
        {\scriptsize \textcolor{blue}{\textbf{Observação:}} Variáveis em azul correspondem às entradas da rede neural, obtidas por medição.}
    \end{minipage}


    \end{multicols}
\end{frame}

\begin{frame}{Estrutura da Rede Neural Proposta}
    \vspace{-0.3cm}
    \begin{figure}
        \centering
        \includegraphics[width=1.0\textwidth]{figures/PIRNN.png} % Insira o caminho correto para a imagem
        \caption{\small Diagrama da arquitetura da PINN.}
    \end{figure}
\end{frame}
\begin{frame}{Função de Loss e Hiperparâmetros da Rede}
    \begin{columns}[T]
        % Coluna da esquerda - Função de perda
        \begin{column}{0.55\textwidth}
            \footnotesize
            A função de perda é composta por quatro parcelas:

            \vspace{0.2cm}
            \scriptsize
            \[
            \text{Loss} = 
            \underbrace{
            \frac{1}{N} \sum_{i=1}^{N} \left( \hat{y}_{d,i}^* - y_{d,i}^* \right)^2
            }_{\text{Mensuráveis diferenciais}} 
            + 
            \underbrace{
            \frac{1}{N} \sum_{i=1}^{N} \left( \hat{y}_{a,i}^* - y_{a,i}^* \right)^2
            }_{\text{Mensuráveis algébricas}} 
            + 
            \underbrace{
            \frac{1}{N} \sum_{i=1}^{N} \left( \frac{d\hat{x}_i}{dt} - \frac{d\hat{x}_i^{an}}{dt} \right)^2
            }_{\text{Consistência com as EDOs}} 
            + 
            \underbrace{
            \frac{1}{N} \sum_{i=1}^{N} \left( \hat{z}_i - z_i \right)^2
            }_{\text{Variáveis algébricas estimadas}}
            \]

            \vspace{0.3cm}
            Onde:
            \begin{itemize}
                \item \( \hat{y}_{d}^* \): variáveis diferenciais mensuráveis (ex: \( \dot{m} \), \( T_P \));
                \item \( \hat{y}_{a}^* \): variáveis algébricas mensuráveis (ex: \( P_2 \), \( T_2 \));
                \item \( \hat{x}_i \): variáveis diferenciais com derivada temporal avaliada numericamente;
                \item \( \hat{z}_i \): variáveis algébricas previstas pela rede;
                \item \( z_i \): variáveis algébricas calculadas pelo solver externo.
            \end{itemize}
        \end{column}

        % Coluna da direita - Hiperparâmetros
        \begin{column}{0.45\textwidth}
            \vspace{2.5cm}
            \begin{block}{\centering Hiperparâmetros do Modelo}
                \footnotesize
                \centering
                \resizebox{\linewidth}{!}{%
                    \begin{tabular}{p{3.6cm} p{2.4cm}}
                        \toprule
                        \textbf{Parâmetro} & \textbf{Valor} \\
                        \midrule
                        Nº de camadas (LSTM) & 1 \\
                        Learning Rate inicial & $1 \cdot 10^{-4}$ \\
                        Tamanho do mini batch & 64 \\
                        Neurônios por camada & 100 \\
                        Nº de épocas & 200 \\
                        Otimizador & Adam \\
                        \bottomrule
                    \end{tabular}%
                }
            \end{block}
        \end{column}
    \end{columns}
\end{frame}


% --- SLIDE 1 ---
\begin{frame}{Resultados de Previsão – Parte 1}
\scriptsize

\begin{center}
    \includegraphics[width=0.85\textwidth]{figures/dados1.png} \\
    \includegraphics[width=0.85\textwidth]{figures/dados2.png}
\end{center}

{\centering
\tiny
\textbf{Figuras 3 e 4:} Comparação entre os valores previstos pelo modelo e os dados reais para as variáveis associadas ao plenum (\(T_P\), \(V_P\), \(P_P\)), saída do compressor (\(P_2\)) e impelidor (\(T_{imp}\), \(V_{imp}\)).
}

\end{frame}

% --- SLIDE 2 ---
\begin{frame}{Resultados de Previsão – Parte 2}
\scriptsize

\begin{center}
    \includegraphics[width=0.85\textwidth]{figures/dados3.png} 
    \includegraphics[width=0.85\textwidth]{figures/dados4.png}
\end{center}



{\centering
\tiny
\textbf{Figuras 5 e 6:} Comparação entre os valores previstos pelo modelo e os dados reais para as variáveis do difusor ($T_{dif}$, $V_{dif}$, $T_2$), do plenum ($\dot{m}$), e da pós-compressão isentrópica ($T_{2s}$, $V_{2s}$), além da saída do compressor ($P_2$).
}

\end{frame}



\begin{frame}{Distribuição do Tempo de Simulação}
\centering
\includegraphics[width=0.65\textwidth]{figures/hist.png}

\vspace{0.3cm}
{\scriptsize \textbf{Figura 5}: Distribuição do tempo de simulação dos experimentos/modelos.}
\end{frame}




