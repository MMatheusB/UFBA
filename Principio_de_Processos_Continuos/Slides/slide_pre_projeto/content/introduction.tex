\section{Introdução}
\begin{frame}{Imagem do Canal}
    \centering
    \includegraphics[width=0.6\textwidth]{Figures/math2sim.jpg}\\[0.3cm]
    \small Imagem do canal \textit{Math2Sim}
\end{frame}
\begin{frame}{Introdução}
\begin{block}{Ideia do Projeto}
O canal \textbf{\textit{Math2Sim}} foi criado como parte da disciplina \textit{Princípios de Processos Contínuos}, com o objetivo de tornar o aprendizado de \textbf{modelagem e simulação} mais atrativo e intuitivo.
\end{block}

\begin{block}{Proposta}
Utilizar \textbf{animações e simulações em Python} para representar fenômenos de engenharia de forma visual e didática.
\end{block}
\end{frame}

% OBJETIVO -----------------------------------------------------------

\begin{frame}{Objetivo do Projeto}
\begin{block}{\faBullseye \quad Objetivo Geral}
Apresentar conteúdos sobre modelagem matemática e física de forma \textbf{visual e acessível}, com foco em sistemas contínuos e fenômenos de engenharia.
\end{block}

\begin{block}{\faCogs \quad Abordagem}
\begin{itemize}
    \item Uso de ferramentas computacionais para visualização de modelos;
    \item Simulação e animação de resultados coerentes;
    \item Apoio à compreensão de conceitos teóricos.
\end{itemize}
\end{block}
\end{frame}