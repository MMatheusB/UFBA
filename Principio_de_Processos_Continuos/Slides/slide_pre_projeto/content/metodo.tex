\section{Estrutura da Rede Neural Artificial}

\begin{frame}{Metodologia Proposta}

\begin{block}{Etapa 1 – Roteiro}
Desenvolvido a partir do conteúdo das aulas e do livro \textit{Modelagem Matemática} de Cláudio Garcia.
\end{block}

\begin{block}{Etapa 2 – Animações}
Criação das animações utilizando a biblioteca \textbf{Manim} em Python.
\end{block}

\begin{block}{Etapa 3 – Edição}
Montagem e edição dos vídeos no \textbf{Adobe Premiere Pro}, incluindo narração e efeitos.
\end{block}

\begin{block}{Etapa 4 – Publicação}
Publicação dos vídeos no \textbf{YouTube}, com acesso livre ao público.
\end{block}

\end{frame}

\begin{frame}{Exemplo de Aplicação do Manim}
    Este vídeo demonstra como o \textbf{Manim} pode ser utilizado para:
    \begin{itemize}
        \item Animar a evolução de sistemas dinâmicos
        \item Visualizar variáveis físicas ao longo do tempo
        \item Facilitar a apresentação de resultados de simulações
    \end{itemize}
\end{frame}

\begin{frame}{Sistema de Compressão de Gás}
    O sistema pode ser descrito pelas seguintes equações governantes:
    \[
    \dot{m} = \frac{A_1}{L_c} \Big(\phi(N,m) P_1 - P \Big), \quad
    \dot{P} = \frac{C^2}{2} \Big( m - \alpha k_v \sqrt{P - P_\text{out}} \Big)
    \]
    Onde:
    \begin{itemize}
        \item $m$ = vazão mássica
        \item $P$ = pressão
        \item $\phi(N,m)$ = função de lookup do compressor
        \item $\alpha$ = abertura da válvula
        \item $k_v$ = constante do sistema
    \end{itemize}
\end{frame}

% Slide 2: Link para o vídeo
\begin{frame}{Evolução do Sistema}
    Para ver a evolução dos gráficos em tempo real, acesse o vídeo:
    
    \vspace{0.5cm}
    \href{https://youtu.be/ruMYnqg-1yk}{\color{blue}\underline{Clique aqui para assistir ao vídeo}}
\end{frame}