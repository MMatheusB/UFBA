\section{Etapas Concluídas}

\begin{frame}{Etapas Concluídas}
\begin{block}{\faTasks \quad Progresso do Projeto}
\begin{enumerate}
    \item Planejamento do conteúdo e roteirização dos vídeos;
    \item Criação das primeiras simulações em Python;
    \item Produção de vídeos sobre:
    \begin{itemize}
        \item Vaso pressurizado de gás;
        \item Circuito série RC e RLC;
        \item Sistema mecânico massa–mola–amortecedor.
    \end{itemize}
    \item Edição e desenvolvimento da identidade visual;
    \item Publicação dos vídeos para teste e feedback.
\end{enumerate}
\end{block}
\end{frame}

% FERRAMENTAS --------------------------------------------------------
\section{Ferramentas Utilizadas}

\begin{frame}{Ferramentas Utilizadas}
\begin{block}{\faCode \quad Desenvolvimento}
\begin{itemize}
    \item \textbf{Python} – Linguagem principal;
    \item \textbf{Manim} – Criação das animações matemáticas;
    \item \textbf{SciPy} – Resolução de equações e simulações.
\end{itemize}
\end{block}

\begin{block}{\faEdit \quad Pós-Produção}
\begin{itemize}
    \item \textbf{Adobe Premiere} – Edição e montagem dos vídeos;
    \item \textbf{GitHub} – Versionamento e organização do projeto.
\end{itemize}
\end{block}
\end{frame}

% DESAFIOS -----------------------------------------------------------
\section{Desafios e Aprendizados}

\begin{frame}{Desafios e Aprendizados}
\begin{block}{\faExclamationTriangle \quad Desafios}
\begin{itemize}
    \item Equilibrar rigor técnico e linguagem acessível;
    \item Aprender o uso avançado do \textit{Manim};
    \item Criar uma identidade visual consistente;
    \item Construir roteiros adequados ao público-alvo.
\end{itemize}
\end{block}

\begin{block}{\faCheckCircle \quad Aprendizados}
Desenvolvimento de habilidades em programação, animação científica, roteiro e comunicação visual aplicada ao ensino de engenharia.
\end{block}
\end{frame}