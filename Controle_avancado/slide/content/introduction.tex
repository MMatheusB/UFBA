\section{Introdução}

\begin{frame}{Contextualização – Linguagem Manim}

\begin{block}{\faUser \quad Criação}
A linguagem \textbf{Manim} (Mathematical Animation Engine) foi criada por \textbf{Grant Sanderson}, conhecido pelo canal \href{https://www.youtube.com/@3blue1brown}{3Blue1Brown}, com o objetivo de produzir vídeos educacionais com animações matemáticas dinâmicas e intuitivas.
\end{block}

\begin{block}{\faUsers \quad Continuidade}
Com o sucesso dos vídeos, a comunidade open-source passou a contribuir com o projeto, originando a \textbf{Manim Community Edition (ManimCE)}, que segue em constante evolução com novas funcionalidades e melhorias.
\end{block}

\begin{block}{\faRocket \quad Popularização}
Hoje, Manim é amplamente utilizado por educadores, estudantes e divulgadores científicos ao redor do mundo, tornando-se uma ferramenta essencial para visualizações matemáticas, científicas e didáticas.
\end{block}

\end{frame}

\begin{frame}{Objetivo do Trabalho}

\begin{block}{\faBullseye \quad Objetivo}
Utilizar o Manim como ferramenta de apoio ao ensino de engenharia, por meio de vídeos educativos aplicados ao contexto de \textbf{processos contínuos}.
\end{block}

\end{frame}